\section{Séance 5 - Réglage de la sensibilité}
\subsection{Objectif}
Durant la séance 4, la valeur du déphasage a été calculée pour optimiser la linéarité
de la réponse du système. Il a également été possible de calculer la sensibilité du système
à ce déphasage.
\vspace{0.2cm}

Le but de cette séance sera donc de modifier le gain du système pour obtenir une sensibilité de
10 V/mm. L'offset du système sera ensuite ajusté pour obtenir une tension de 0 V à la position
de repos.
\vspace{0.2cm}

\subsection{Théorie}
Pour obtenir la sensibilité souhaitée, il convient de modifier le gain du filtre passe bas
de notre système.

\begin{figure}[H]
    \centering
    \includegraphics[width=10cm]{Images/Seance5/Gain système.png}
    \caption{Circuit du filtre passe bas}
    \label{fig:FPB}
\end{figure}

La résistance $R_{14}$ ayant une valeur de 0 $\Omega$, le gain de ce système est défini comme suit :

\begin{equation*}
     |K| = \frac{R_{10}}{R_{13}}
\end{equation*}

Il est cependant également possible d'écrire le gain en fonction de l'entrée et de la sortie du
système :

\begin{equation*}
    |K| = \frac{S_{out}}{S_{in}}
\end{equation*}

Connaissant la sensibilité d'entrée ainsi que celle voulue en sortie, il est possible, en fixant
la valeur d'une résistance, de calculer la valeur de la seconde. La valeur de R$_{10}$ a été fixée
à 10 k$\Omega$. Il est donc possible de calculer le tableau suivant:


\begin{table}[H]
    \centering
    \begin{tabular}{|l|rl|}
    \cline{1-3}
    S$_{in}$  & 0,3023    & V/mm    \\ \hline
    S$_{out}$ & 10         & V/mm    \\ \hline
    K    & 33,08 &  \\ \hline
    R$_{10}$  & 10      & k$\Omega$     \\ \hline
    R$_{13}$  & 302,3     & $\Omega$     \\ \hline
    \end{tabular}
    \caption{Résumé des valeurs calculées}
    \label{tab:ResumeValeurs}
    \end{table}

Pour l'ajustement de l'offset, il sera uniquement fait de manière expérimentale.

\subsection{Manipulation}
Dans un premier temps, deux résistance ont été mises en série
\subsection{Analyse}

\subsection{Conclusion}