\section{Séance 5 - Réglage de la sensibilité}
\subsection{Objectif}
Durant la séance 4, la valeur du déphasage a été calculée pour optimiser la linéarité
de la réponse du système. Il a également été possible de calculer la sensibilité du système
à ce déphasage.
\vspace{0.2cm}

Le but de cette séance sera donc de modifier le gain du système pour obtenir une sensibilité de
10 V/mm. L'offset du système sera ensuite ajusté pour obtenir une tension de 0 V à la position
de repos.
\vspace{0.2cm}

\subsection{Théorie}
Pour obtenir la sensibilité souhaitée, il convient de modifier le gain du filtre passe bas
de notre système.

\begin{figure}[H]
    \centering
    \includegraphics[width=10cm]{Images/Seance5/Gain système.png}
    \caption{Circuit du filtre passe bas}
    \label{fig:FPB}
\end{figure}

La résistance $R_{14}$ ayant une valeur de 0 $\Omega$, le gain de ce système est défini comme suit :

\begin{equation*}
     |K| = \frac{R_{10}}{R_{13}}
\end{equation*}

Il est cependant également possible d'écrire le gain en fonction de l'entrée et de la sortie du
système :

\begin{equation*}
    |K| = \frac{S_{out}}{S_{in}}
\end{equation*}

Connaissant la sensibilité d'entrée ainsi que celle voulue en sortie, il est possible, en fixant
la valeur d'une résistance, de calculer la valeur de la seconde. La valeur de R$_{10}$ a été fixée
à 10 k$\Omega$. Il est donc possible de calculer le tableau suivant :


\begin{table}[H]
    \centering
    \begin{tabular}{|l|rl|}
    \cline{1-3}
    S$_{in}$  & 0,3023    & V/mm    \\ \hline
    S$_{out}$ & 10         & V/mm    \\ \hline
    K    & 33,08 &  \\ \hline
    R$_{10}$  & 10      & k$\Omega$     \\ \hline
    R$_{13}$  & 302,3     & $\Omega$     \\ \hline
    \end{tabular}
    \caption{Résumé des valeurs calculées}
    \label{tab:ResumeValeurs}
    \end{table}

Pour l'ajustement de l'offset, il sera uniquement fait de manière expérimentale.

\subsection{Manipulation}
Dans un premier temps, deux résistances ont été mises en série afin d'obtenir la résistance
R$_{13}$. Cette nouvelle résistance a été installée sur l'emplacement R$_{13}$

\begin{table}[H]
    \centering
    \begin{tabular}{|l|r|}
    \cline{1-2}
    R$_{1}$   & 270 $\Omega$   \\ \hline
    R$_{2}$   & 32  $\Omega$        \\ \hline
    R$_{1+2}$ & 302 $\Omega$ \\ \hline
    R$_{13_{mes}}$ & 298.7 $\Omega$ \\ \hline
    \end{tabular}
    \caption{Résumé des valeurs calculées}
    \label{tab:ResumeValeurs}
\end{table}

Le gain réel sera donc:
\begin{equation*}
    |K_{reel}| = \frac{R_{10}}{R_{13_{mes}}} = 33.48   
\end{equation*}

Qui donnera une sensibilité réelle en sortie de conditionneur de:

\begin{equation*}
    S_{out} = |K_{reel}|\cdot S_{in} =10.12 \text{ V/mm} 
\end{equation*}
\vspace{0.2cm}

Le jumper J6 a été enlevé puis le potentiomètre a été utilisé pour régler l'offset sur la tension 
de sortie. Pour une distance à la cible de 0.7 mm, une tension de sortie de 0 V est souhaitée.
\vspace{0.2cm}

La tension de sortie a ensuite été mesurée à toutes les positions, pour un déphasage de 0°.
\begin{figure}[H]
    \centering
    \includegraphics[width=15cm]{Images/Seance5/uout1.png}
    \caption{Tension de sortie à déphasage nul}
    \label{fig:uout1}
\end{figure}

En première instance, il est possible d'observer que la tension de sortie n'est pas linéaire en
fonction de la distance, contrairement à ce qui est attendu. 

\vspace{0.2cm}

En remesurant des valeurs de tension en fonction de la distance à la cible pour plusieurs déphasages,
il a été possible d'observer un comportement qui semblait linéaire pour un déphasage de 120°.

\begin{figure}[H]
    \centering
    \includegraphics[width=15cm]{Images/Seance5/uout2.png}
    \caption{Tension de sortie à déphasage 120°}
    \label{fig:uout2}
\end{figure}

Le gain ainsi que la résistance $R_{13}$ ont ensuite été recalculés pour les valeurs mesurées :
\begin{equation*}
    |K_2| = \frac{10}{0.931} = 10.74
\end{equation*}
\begin{equation*}
    R_{13} = \frac{R_{10}}{|K_2|}= \frac{10^4}{10.74}= 931\text{ }\Omega
\end{equation*}

Par manque de rigueur et une volonté de vitesse, une résistance d'une valeur de 1 k$\Omega$
a été utilisée pour $R_{13}$. Le gain sera donc de 10 et la sensibilité devrait être de -9.31 V/mm.
\vspace{0.2cm}

Lors du réglage de l'offset, le potentiomètre ne permettant pas d'atteindre une tension de 0 V à
0.7 mm, 180° ont été ajoutés au déphasage. Cette manipulation ne devrait pas changer le comportement
du système autrement que de multiplier la sensibilité par -1, puisque le système est $\pi$ périodique.
L'offset a ensuite pu être compensé.
\vspace{0.2cm}

La sensibilité devrait donc être de 9.31 V/mm.

\begin{figure}[H]
    \centering
    \includegraphics[width=15cm]{Images/Seance5/uout3.png}
    \caption{Tension de sortie à déphasage 300°}
    \label{fig:uout3}
\end{figure}

Bien qu'il soit possible de voir que la tension réelle en fonction de la position n'est pas 
parfaitement linéaire, cette dernière, à contrario de la première itération, s'en approche assez
pour être utilisable.
\subsection{Analyse}

\subsection{Conclusion}