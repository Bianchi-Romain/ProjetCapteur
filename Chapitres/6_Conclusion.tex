\newpage
\section{Conclusion}

Ces trois premières séances ont permis une première étude du capteur
inductif.

Elles ont permis de :

\begin{itemize}
    \item Déterminer le matériau de la cible du capteur comme étant du cuivre;
    \item Déterminer la fréquence de résonance du circuit de la carte et déterminer la fonction 
    de transfert correspondante, qui s'est avérée correspondre au résultat attendu; 
    \item Décrire et observer le comportement du multiplicateur et du filtre RC. 
\end{itemize}

\textbf{Améliorations :}

Bien que les résultats soient globalement bons, ces trois séances ont eu leurs lots de surprises.
Le capteur initialement utilisé présentait un défaut de câblage ce qui a nécessité une reprise 
complète des mesures de la séance 1 et engendré un certain retard. Une erreur d'anticipation et  
une réaction trop lente de notre part qui aurait pu être évitée.



