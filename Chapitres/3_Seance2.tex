\input{Chapitres/ch3Py.tex}

\section{Séance 2 - Fréquence de résonance et fonction de transfert}


\subsection{Objectifs}
La première séance de ce projet a permis les observations suivantes :
\begin{itemize}
    \item La fréquence idéale du système se situe à 6.78 MHz (Figure \ref{fig: S}).
    \item À la fréquence de 2.6 MHz, la résistance étant constante en fonction de la position,
    il est difficile d'avoir un capteur de position avec ce système.
    \item Au-delà de la fréquence de 2.6 MHz, le modèle théorique devient trop différent de la réalité.
    \item À une fréquence proche de 150 kHz, pour l'inductance et la résistance, il est possible
    de considérer qu'entre deux positions, le système se comporte de manière linéaire.  
\end{itemize}

\vspace{0,2cm}

Le but de cette séance sera donc d'abaisser la fréquence de résonance du système à une valeur
exploitable. L'abaissement de la fréquence de résonance se fait par la mise en parallèle d'une 
capacité.
\vspace{0,2cm}

Cette capacité étant fournie, la fréquence de résonance sera déterminée théoriquement puis par
manipulation. Enfin, une résistance sera ajoutée pour maximiser la sensibilité du système. 


\subsection{Méthode}


\subsection{Résultats}

\subsection{Analyse}

\subsection{Conclusion}
