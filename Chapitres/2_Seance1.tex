\begin{pycode}



\end{pycode}

\section{Séance 1 - Caractéristique de la bobine }

\subsection{Objectifs}

Le but de cette première séance est de caractériser le capteur de proximité inductif à travers 
l’analyse de la résistance et l'inductance de la bobine, en fonction de la distance cible-capteur 
 et de la fréquence. 

\subsection{Méthode}

\begin{enumerate}
    \item Brancher la maquette sur l'analyseur d'impédance comme montré ci-dessous (cf. fig. \ref{fig: AI});
    \item Déterminer la distance mécanique initiale et se positionner à 0.5 mm de celle-ci ;
    \item Enregistrer la mesure de l'appareil qui permet d'obtenir l'impédance ainsi que l'inductance de 
    la bobine sur un intervalle de 150 kHz à 15 MHz ;
    \item Répéter l'opération par pas de 0.05 mm jusqu'à atteindre 0.90 mm ce qui totalise 9 mesures.
\end{enumerate}

\insererfigure{Images/Seance1/AnalyseurImpedance.jpg}{5cm}{analyseur d'impédance}{AI}

\subsection{Caractérisation de la bobine}

\subsubsection{Impédance}


Les mesures réalisées avec l'analyseur d'impédance ont été sauvegardées en un total
de 9 fichiers CSV, un pour chacune des distances. Chaque fichier comporte une colonne propre à
la fréquence, l'inductance et la résistance. Ainsi, il est possible de déterminer l'impédance relative
à la fréquence selon les 9 positions de la bobine.\\


Impédance de la bobine : 

\begin{equation}
    Z_b = R_b + j \omega L_b
\end{equation}

Avec : \\
$Z_b$, l'impédance de la bobine [$\Omega$];\\
$R_b$, la résistance de la bobine [$\Omega$];\\
$L_b$, l'inductance de la bobine;\\
$\omega$, la pulsation = $ 2 \cdot \pi \cdot f$ [rad/s] ;\\
f, la fréquence [Hz].\\

L'impédance ainsi déterminée a permis de représenter graphiquement l'inductance et la résistance de 
la bobine en fonction de la fréquence :


\insererfigure{Images/Seance1/Lf.png}{8cm}{Inductance en fonction de la fréquence}{Lf}
\insererfigure{Images/Seance1/Rf.png}{8cm}{Résistances en fonction de la fréquence}{Rf}

L'allure des courbes représentées sur les graphiques (cf. fig. \ref{fig: Lf} et \ref{fig: Rf})
correspondent au comportement d'un circuit RLC. Ce résultat est représentatif du montage étudié.
En effet, la \textbf{R}ésistance et est celle du fil de la bobine, \textbf{L}'inductance est celle 
de la bobine, le \textbf{C}ondensateur est la capacité relative à la 
cible et à sa distance.\\
La fréquence de résonance du circuit peut-être aisément relevée puis-ce qu'il s'agit du passage à 0 
de l'inductance et le pic de résistance. On relève graphiquement à la 
position de repos une fréquence de résonance de 6.77 Mhz.\\

L'inductance et la résistance en fonction de la distance ont ensuite été représentés :

\insererfigure{Images/Seance1/Lx.png}{8cm}{Inductance en fonction de la distance}{Lx}
\insererfigure{Images/Seance1/Rx.png}{8cm}{Résistances en fonction de la distance}{Rx}

Les courbes tracées permettent de déterminer le type de matériau de la cible, ferromagnétique ou 
non-ferromagnétique. Les figures \ref{fig: Lx} et \ref{fig: Rx} montrent une augmentation de 
l'inductance et une diminution de la résistance lors de l'éloignement de la cible ce qui correspond 
au comportement d'un matériau conducteur non-ferromagnétique. 
De plus, la cible est de couleur "orangée", ce qui permet de conclure qu'elle est en \textbf{cuivre}.\\  

Les régressions réalisées (cf. fig. \ref{fig: Lx} et \ref{Rx}) 



\insererfigure{Images/Seance1/Zx.png}{8cm}{Impédance de la bobine aux distances}{Zx}

\subsubsection{Sensibilité}

Grâce aux valeurs relevées, il est également possible de déterminer la sensibilité de
ce capteur qui, pour une position de repos à 0.7 mm s'exprime : 

\begin{equation}\label{eq:s}
    \left . S_b \right |_{x=0.7} = \left . \frac{\left | \delta \underline{Z_b}\right |}{\delta x} \right |_{x=0.7}
\end{equation}

Avec: \\
$S_b$, la sensibilité [$\Omega/mm$];\\
$\delta Z$, l'écart d'impédance = $Z_{0.75} - Z_{0.65}$ [$\Omega$];\\
$\delta x$, l'écart de distance  = $x_{0.75} - x_{0.65}$ [mm].

\insererfigure{Images/Seance1/S.png}{8cm}{Sensibilité en fonction de la fréquence}{S}

Le graphique ci-dessus montre la sensibilité calculée avec l'équation \ref{eq:s} en fonction
de la fréquence, pour une position de repos de 0.7mm. Cette représentation permet d'observer
qu'un pic sensibilité se forme de 5 à 9 MHz, culminant à 21'235 [$\Omega$/mm] à la fréquence 
de 6.78 MHz. 

 

\subsection{Analyse}

\subsection{Conclusion}

