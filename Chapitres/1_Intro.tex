\begin{pycode}
    
\end{pycode}

\section{Introduction}

Dans le cadre du cours "Capteurs", trois séances de laboratoires portant
sur l'étude du comportement d'un capteur inductif à courant de Foucault ont été effectuées. 
Ce document présent est un rapport intermédiaire de ce mini-projet répondant aux objectifs.

\subsection{Objectifs}

\begin{itemize}
    \item Appliquer les connaissances acquises au cours ;
    \item Observer comment se comportent la résistance et l'inductance du capteur ;
    \item Optimiser la sensibilité du capteur ;
    \item Déterminer la fonction de transfert et la transformée de Möbius. 
\end{itemize}

\subsection{Matériel}

\begin{table}[H]
    \centering
    \begin{tabular}{|c|c|c|c|}
    \hline
    \textbf{Nom}          & \textbf{Marque} & \textbf{Modèle} & \textbf{N° de série} \\ \hline
    Capteur (maquette)    & -               & HEIG-VD         & 09                   \\ \hline
    Circuit imprimé       & -               & HEIG-VD         & 12                   \\ \hline
    Boite de laboratoire  & -               & HEIG-VD         & 1                    \\ \hline
    Multimètre            & Keysight        & 34460A          & MY53102179           \\ \hline
    Oscilloscope          & Tektronix       & DPO 2014B       & C030007              \\ \hline
    Générateur de signal  & SIGLENT         & SDG2082X        & SDG"XCA1160916       \\ \hline
    Générateur de tension & GWINSTEK        & GPD-3303S       & A140629              \\ \hline
    Analyseur d'impédance & Keysight        & E4990A          & MY54100421           \\ \hline
    \end{tabular}
    \caption{Liste du matériel}
    \end{table}





    

