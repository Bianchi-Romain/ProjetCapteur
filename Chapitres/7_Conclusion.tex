\newpage
\section{Conclusion}

Ce travail a permis une étude du capteur inductif.

Il a notament permis de :

\begin{itemize}
    \item Déterminer le matériau de la cible du capteur comme étant du cuivre;
    \item Déterminer la fréquence de résonance du circuit de la carte et déterminer la fonction 
    de transfert correspondante, qui s'est avérée correspondre au résultat attendu; 
    \item Décrire et observer le comportement du multiplicateur et du filtre RC. 
    \item Observer les différences de linéarité selon le déphasage appliqué
    \item Déterminer le déphasage optimal
    \item Écrire des spécifications pour un utilisateur potentiel
\end{itemize}

\textbf{Améliorations :}

Bien que les résultats soient globalement bons, chaque séance a eu son lot de surprises.
Le capteur initialement utilisé présentait un défaut de câblage ce qui a nécessité une reprise 
complète des mesures de la séance 1 et engendré un certain retard. Les valeurs obtenues lors de la
séance 4 étaient aberrantes. Ayant un semblant de cohérence dans ces valeurs, le problème n'a été
observé que durant la séance 5. La cause de ce problème reste inconnue et à explorer. 
\vspace{0.2cm}

L'incertitude du capteur peut être amélioré en déterminant correctement le déphasage optimal.
\vspace{0.2cm}

\vspace{0.2cm}
Une erreur d'anticipation et une réaction trop lente de notre part aurait pu éviter certains des
problèmes rencontrés.
